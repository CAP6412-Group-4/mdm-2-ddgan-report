% CVPR 2023 Paper Template
% based on the CVPR template provided by Ming-Ming Cheng (https://github.com/MCG-NKU/CVPR_Template)
% modified and extended by Stefan Roth (stefan.roth@NOSPAMtu-darmstadt.de)

\documentclass[10pt,twocolumn,letterpaper]{article}

%%%%%%%%% PAPER TYPE  - PLEASE UPDATE FOR FINAL VERSION
\usepackage[final]{cvpr}      % To produce the REVIEW version
% \usepackage{cvpr}              % To produce the CAMERA-READY version
% \usepackage[pagenumbers]{cvpr} % To force page numbers, e.g. for an arXiv version

% Include other packages here, before hyperref.
\usepackage{graphicx}
\usepackage{amsmath}
\usepackage{amssymb}
\usepackage{booktabs}
\usepackage[none]{hyphenat}


% It is strongly recommended to use hyperref, especially for the review version.
% hyperref with option pagebackref eases the reviewers' job.
% Please disable hyperref *only* if you encounter grave issues, e.g. with the
% file validation for the camera-ready version.
%
% If you comment hyperref and then uncomment it, you should delete
% ReviewTempalte.aux before re-running LaTeX.
% (Or just hit 'q' on the first LaTeX run, let it finish, and you
%  should be clear).
\usepackage[pagebackref,breaklinks,colorlinks]{hyperref}


% Support for easy cross-referencing
\usepackage[capitalize]{cleveref}
\crefname{section}{Sec.}{Secs.}
\Crefname{section}{Section}{Sections}
\Crefname{table}{Table}{Tables}
\crefname{table}{Tab.}{Tabs.}


%%%%%%%%% PAPER ID  - PLEASE UPDATE
\def\cvprPaperID{*****} % *** Enter the CVPR Paper ID here
\def\confName{CVPR}
\def\confYear{2023}


\begin{document}

% <><><><><><><><><><><><><><><><><><><><><><><><><><><><><><><><><><><><><><><><>
%                             TITLE AND AUTHORS
% <><><><><><><><><><><><><><><><><><><><><><><><><><><><><><><><><><><><><><><><>
\title{Motion Diffusion Model to Denoising Diffusion GAN: Efficient Motion Sampling}

\author{
    Ronald Campos\\
    University of Central Florida\\
    4000 Central Florida Blvd, Orlando, FL 32816\\
    {\tt\small roncamposj@knights.ucf.edu}
    \and
    Muhammad Asad Haider\\
    University of Central Florida\\
    4000 Central Florida Blvd, Orlando, FL 32816\\
    {\tt\small haider24@knights.ucf.edu}
    \and
    Suneet Tipirneni\\
    University of Central Florida\\
    4000 Central Florida Blvd, Orlando, FL 32816\\
    {\tt\small suneet.tipirneni@knights.ucf.edu}
    \and
    Stefan Werleman\\
    University of Central Florida\\
    4000 Central Florida Blvd, Orlando, FL 32816\\
    {\tt\small stefanwerleman@knights.ucf.edu}
}


\maketitle

% <><><><><><><><><><><><><><><><><><><><><><><><><><><><><><><><><><><><><><><><>
%                                  ABSTRACT
% <><><><><><><><><><><><><><><><><><><><><><><><><><><><><><><><><><><><><><><><>
\begin{abstract}
    All existing motion diffusion models use the standard diffusion process which yields high 
    quality samples. However, the standard process for these models can be inefficient. These are one of
    the challenges with the learning trilemma and this works concerns embedding an existing motion diffusion 
    model into denoising diffusion GANs to create a hybrid architecture of the motion diffusion model. This 
    new hybrid model will satisfy the learning trilemma, thus improving the sampling speed when training and generating a motion sample.
    \url{https://github.com/CAP6412-Group-4/denoising-diffusion-gan}
\end{abstract}


% <><><><><><><><><><><><><><><><><><><><><><><><><><><><><><><><><><><><><><><><>
%                                  INTRODUCTION
% <><><><><><><><><><><><><><><><><><><><><><><><><><><><><><><><><><><><><><><><>
\section{Introduction}
\label{sec:intro}

\subsection{Human Motion Diffusion Model}

\subsection{Improving Sampling}

\subsection{Integrating Motion Diffusion Model Into DDGAN}


% <><><><><><><><><><><><><><><><><><><><><><><><><><><><><><><><><><><><><><><><>
%                                  RELATED WORK
% <><><><><><><><><><><><><><><><><><><><><><><><><><><><><><><><><><><><><><><><>
\section{Related Work}
\label{sec:related-work}

\subsection{Human Motion Diffusion Model}

\subsection{Denoising Diffusion GANs}


% <><><><><><><><><><><><><><><><><><><><><><><><><><><><><><><><><><><><><><><><>
%                                  METHOD
% <><><><><><><><><><><><><><><><><><><><><><><><><><><><><><><><><><><><><><><><>
\section{Method}
\label{sec:method}

\subsection{Motion Diffusion Model Integration}

\subsection{Adapting The Loss}

\subsection{Training}

% <><><><><><><><><><><><><><><><><><><><><><><><><><><><><><><><><><><><><><><><>
%                                  EXPERIMENTS
% <><><><><><><><><><><><><><><><><><><><><><><><><><><><><><><><><><><><><><><><>
\section{Experiments}
\label{sec:experiments}

% <><><><><><><><><><><><><><><><><><><><><><><><><><><><><><><><><><><><><><><><>
%                                  RESULTS
% <><><><><><><><><><><><><><><><><><><><><><><><><><><><><><><><><><><><><><><><>
\section{Results}
\label{sec:results}

\subsection{Quantitative Results}

\subsection{Qualitative Results}

% <><><><><><><><><><><><><><><><><><><><><><><><><><><><><><><><><><><><><><><><>
%                           ADDITIONAL APPLICATIONS
% <><><><><><><><><><><><><><><><><><><><><><><><><><><><><><><><><><><><><><><><>
\section{Additional Applications}
\label{sec:additional-applications}

Once our hybrid motion diffusion model was complete and generating results, we want to see how we can
leverage the motion samples for other applications. One application we wanted to try was person image 
synthesis.

\subsection{Using Motion Samples for Person Image Synthesis}



% <><><><><><><><><><><><><><><><><><><><><><><><><><><><><><><><><><><><><><><><>
%                         CONCLUSION AND FUTURE WORKS
% <><><><><><><><><><><><><><><><><><><><><><><><><><><><><><><><><><><><><><><><>
\section{Conclusion and Future Works}
\label{sec:conlusion}

% <><><><><><><><><><><><><><><><><><><><><><><><><><><><><><><><><><><><><><><><>
%                               REFERENCES
% <><><><><><><><><><><><><><><><><><><><><><><><><><><><><><><><><><><><><><><><>
{\small
\bibliographystyle{ieee_fullname}
\bibliography{egbib}
}


\end{document}
